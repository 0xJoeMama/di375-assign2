\documentclass[11pt, a4paper]{report}

\usepackage{fontspec}
\setmainfont[Script=Greek]{GFS Artemisia}
\setmonofont{FreeMono}

\usepackage{polyglossia}
\setdefaultlanguage{greek}
\setotherlanguages{english}

\usepackage{geometry}
\geometry{margin=3cm}

\usepackage{karnaugh-map}
\usepackage{amsmath}
\usepackage{circuitikz}
\usepackage{minted2}

\begin{document}

% header section
\author{Πλαστήρας Πέτρος}
\title{Εργασία \#2 - Μέρος Α}
\date{\today}
\maketitle

\section{Ερώτημα 1}
Έστω ο κάτωθι πίνακας αληθείας για κύκλωμα με 3 εισόδους $x,y,z$
και έξοδο $f$:

\begin{center}
	\begin{tabular} { | c | c | c | c |}
		\hline
		$x$ & $y$ & $z$ & $f$ \\
		\hline
		0   & 0   & 0   & 0   \\
		0   & 0   & 1   & 0   \\
		0   & 1   & 0   & 1   \\
		0   & 1   & 1   & 1   \\
		1   & 0   & 0   & 0   \\
		1   & 0   & 1   & 1   \\
		1   & 1   & 0   & 1   \\
		1   & 1   & 1   & 1   \\
		\hline
	\end{tabular}
\end{center}

Μπορούμε να σχεδιάσουμε τον χάρτη Karnaugh που αντιστοιχεί στον παραπάνω πίνακα αληθείας ως εξής:

\begin{center}
\begin{karnaugh-map}[4][2][1][$x$][$y$][$z$]
\minterms{2, 3, 5, 6, 7}
\maxterms{0,1,4}
\end{karnaugh-map}
\end{center}

Τώρα με την χρήση του παραπάνω χάρτη Karnaugh, μπορούμε να βρούμε την εξίσωση της $f$.
Αρχικά εντοπίζουμε τους κύκλους του χάρτη όπως φαίνεται παρακάτω:

\begin{center}
\begin{karnaugh-map}[4][2][1][$x$][$y$][$z$]
\minterms{2, 3, 5, 6, 7}
\maxterms{0,1,4}
\implicant{3}{6}
\implicant{5}{7}
\end{karnaugh-map}
\end{center}

Από αυτό προκύπτει η παρακάτω εξίσωση για την $f$:
\begin{align*}
	f & = y + xz
\end{align*}

Μπορούμε να απλοποιήσουμε τον παραπάνω πίνακα αληθείας, επιτρέποντας μόνο στα $y$ και $z$ να μεταβληθούν.
Για να γίνει αυτό ελέγχουμε τις τιμές της $f$ για τις διάφορες τιμές των $y$ και $z$:
\begin{enumerate}
	\item Για $y = 0$ και $z = 0$, ισχύει ότι: $f = y + xz = 0 + 0 \cdot x = 0$.
	\item Για $y = 0$ και $z = 1$, ισχύει ότι: $f = y + xz = 0 + 1 \cdot x = x$.
	\item Για $y = 1$ , ισχύει ότι: $f = y + xz = 1 + xz = 1$, λόγω του θεωρήματος του κυρίαρχου στοιχείου.
\end{enumerate}
Μπορούμε να απλοποιήσουμε τον παραπάνω πίνακα αληθείας, επιτρέποντας μόνο στα $y$ και $z$ να μεταβληθούν
Επιτρέποντας τώρα μόνο στα $y$ και $z$ να μεταβληθούν προκύπτει ο παρακάτω πίνακας αληθείας:

\begin{center}
	\begin{tabular} { | c | c | c |}
		\hline
		$y$ & $z$ & $f$ \\
		\hline
		0   & 0   & 0   \\
		0   & 1   & $x$ \\
		1   & 0   & 1   \\
		1   & 1   & 1   \\
		\hline
	\end{tabular}
\end{center}

Αυτό συμβαίνει διότι αν ισχύει $y = 0$ και $z = 1$ παρατηρούμε ότι η $f$ ταυτίζεται με την τιμή του $x$.
Από αυτό προκύπτει ξανά(ως άθροισμα γινομένων) ότι η $f$ έχει εξίσωση $f = y + yz$.

Για να σχεδιάσουμε το κύκλωμα στην παραπάνω μορφή του, θα χρησιμοποιήσουμε έναν πολυπλέκτη 4 σε 1 για να υλοποιήσουμε τον απλοποιημένο πίνακα αληθείας:

\begin{center}
	\begin{circuitikz}
		\ctikzset{logic ports=ieee}
		\tikzset{mux 4to1/.style={muxdemux, muxdemux def={Lh=5, Rh=3,  NL=4, NT=0, NB=2, NR=1, w=2}, muxdemux label={L1=$00$, L2=$01$, L3=$10$, L4=$11$}}}
		\draw
		(7, 0) node[label] (f) {$f$}
		(f) ++ (-2, 0) node[mux 4to1] (mux) {}
		(mux) ++ (-3, 2) node[vcc] (vcc) {$V_{DD}$}
		(mux) ++ (-2, -3) node[ground] (gnd) {$GND$}

		(mux.lpin 2) ++ (-4, 0) node[label] (X) {$x$}
		(X) ++ (0, -2) node[label] (Y) {$y$}
		(Y) ++ (0, -1) node[label] (Z) {$z$}

		(mux.rpin 1) -- (f)
		(vcc) |- (mux.lpin 3)
		(vcc) |- (mux.lpin 4)
		(gnd) |- (mux.lpin 1)
		(X) -- (mux.lpin 2)
		(Y) -| (mux.bpin 1)
		(Z) -| (mux.bpin 2)

		;
	\end{circuitikz}
\end{center}

Αντίστοιχα μπορούμε να φτιάξουμε τον πίνακα αληθείας, κάνοντάς τον να εξαρτάται μόνο από το $x$. Για να γίνει αυτό πρέπει να βρούμε τις τιμές τις $f$ για τις διάφορες τιμές του $x$:

\begin{enumerate}
	\item Για $x = 0$, έχουμε ότι $f = y + xz = y + 0 \cdot z = y$.
	\item Για $x = 1$, έχουμε ότι $f = y + xz = y + 1 \cdot z = y + z$.
\end{enumerate}

Με βάση τα παραπάνω ο νέος πίνακας αληθείας είναι:
\begin{center}
	\begin{tabular} {|c|c|}
		\hline
		$x$ & $f$     \\
		\hline
		0   & $y$     \\
		1   & $y + z$ \\
		\hline
	\end{tabular}
\end{center}

Μπορούμε τώρα να φτιάξουμε το λογικό κύκλωμα που αντιστοιχεί στον παραπάνω πίνακα αληθείας χρησιμοποιώντας έναν πολυπλέκτη 2 σε 1:

\begin{center}
	\begin{circuitikz}
		\ctikzset{logic ports=ieee}
		\tikzset{mux 2to1/.style={muxdemux, muxdemux def={Lh=4, Rh=2,NL=2, NT=0, NB=1, NR=1, w=2}, muxdemux label={L1=$0$, L2=$1$}}}
		\draw
		(7, 0) node[label] (f) {$f$}
		(f) ++ (-2, 0) node[mux 2to1] (mux) {}

		(mux.bpin 1) ++ (0, -1) node[label] (X) {$X$}
		(mux.lpin 2) ++ (-2, 0) node[ieeestd or port] (yorz) {}
		(mux.lpin 1) ++ (-5, 0) node[label] (y) {$y$}
		(yorz.in 2) ++ (-2, 0) node[label] (z) {$z$}

		(X) |- (mux.bpin 1)
		(y) -- (mux.lpin 1)
		(y) ++ (1, 0)  |- (yorz.in 1)
		(z) -- (yorz.in 2)
		(mux.rpin 1) -- (f)
		(yorz.out) |- (mux.lpin 2)
		;
	\end{circuitikz}
\end{center}

\section{Ερώτημα 2}
Δίνεται ο ακόλουθος κώδικας VHDL.
\begin{enumerate}
	\item Να βρείτε τους πίνακες αληθείας για τα X και Υ.
	\item Να βρείτε τους πίνακες k-map για τα x και y και κατόπιν τις ελαχιστοποιημένες εξισώσεις Boole των X, Υ με βάση τα K-maps.
	\item Να υλοποιήσετε τη συνάρτηση Υ με τη χρήση πολυπλέκτη 4 σε 1, με εισόδους επιλογής B, C.
\end{enumerate}

\inputminted{vhdl}{./code/part-1/prompt.vhdl}

Προκειμένου να βρούμε τους πίνακες αληθείας για τα $X$ και $Y$ θα χρησιμοποιήσουμε της λογική του κώδικα κι όχι δοκιμαστηκές εκτελέσεις.
Στην U1 παρατηρούμε ότι το $Y$ παίρνει νέα τιμή το 1 σε δύο κλάδους.
Αν το $B$ είναι 1 και το $C = D$ και αν το $Α$ είναι 0 και διάφορο του $C$ δηλαδή αν το $C$ είναι 1.
Διαφορετικά είναι 0. Συνεπώς αυτό παραπέμπει στον παρακάτω πίνακα αληθείας:
\begin{center}
	\begin{tabular} {|c|c|c|c|c|}
		\hline
		$A$ & $B$ & $C$ & $D$ & $Y$ \\
		\hline
		0   & 0   & 0   & 0   & 0   \\
		0   & 0   & 0   & 1   & 0   \\
		0   & 0   & 1   & 0   & 1   \\
		0   & 0   & 1   & 1   & 1   \\
		0   & 1   & 0   & 0   & 1   \\
		0   & 1   & 0   & 1   & 0   \\
		0   & 1   & 1   & 0   & 0   \\
		0   & 1   & 1   & 1   & 1   \\
		1   & 0   & 0   & 0   & 0   \\
		1   & 0   & 0   & 1   & 0   \\
		1   & 0   & 1   & 0   & 0   \\
		1   & 0   & 1   & 1   & 0   \\
		1   & 1   & 0   & 0   & 1   \\
		1   & 1   & 0   & 1   & 0   \\
		1   & 1   & 1   & 0   & 0   \\
		1   & 1   & 1   & 1   & 1   \\
		\hline
	\end{tabular}
\end{center}

Παρατηρούμε ότι το $x$ δεν εξαρτάται από την τιμή του $A$ καθώς το process U2 δεν χρησιμοποιεί πουθενά το σήμα $A$.
Συνεπώς, για το $x$ o πίνακας αληθείας είναι ο παρακάτω:
\begin{center}
	\begin{tabular} {|c|c|c|c|}
		\hline
		$B$ & $C$ & $D$ & $X$ \\
		\hline
		0   & 0   & 0   & 0   \\
		0   & 0   & 1   & 1   \\
		0   & 1   & 0   & 0   \\
		0   & 1   & 1   & 0   \\
		1   & 0   & 0   & 0   \\
		1   & 0   & 1   & 0   \\
		1   & 1   & 0   & 0   \\
		1   & 1   & 1   & 1   \\
		\hline
	\end{tabular}
\end{center}

Μπορούμε τώρα αντίστοιχα να σχεδιάσουμε τους χάρτες Karnaugh που αντιστοιχούν στο $X$ και στο $Y$:
\begin{center}
\begin{karnaugh-map}[4][4][1][$A$][$B$][$C$][$D$]
\minterms{2, 3, 4, 6, 7,13,15}
\maxterms{0,1,5,8,9,10,11,12,14}
\implicant{3}{6}
\implicant{13}{15}
\implicantedge{4}{4}{6}{6}
\end{karnaugh-map}
\begin{karnaugh-map}[4][2][1][$B$][$C$][$D$]
\minterms{1, 7}
\maxterms{0, 2, 3, 4, 5, 6}
\implicant{1}{1}
\implicant{7}{7}
\end{karnaugh-map}
\end{center}

Σύμφωνα με τα παραπάνω προκύπτουν οι παρακάτω εξισώσεις για τα $X$ και $Y$:
\begin{align*}
	Y & = B \cdot \bar{D} + C \cdot D \cdot A + \bar{D} \cdot C \cdot \bar{A} \\
	  & = B \cdot \bar{D} + C \cdot (A \cdot D + \bar{A} \cdot \bar{D})       \\
	  & = B \cdot \bar{D} + C \cdot \overline{D \oplus A}                     \\
	X & = \bar{C} \cdot B \cdot \bar{D} + C \cdot B \cdot D                   \\
	  & = B \cdot (\bar{C} \cdot \bar{D} + C \cdot D)                         \\
	  & = B \cdot \overline{C \oplus D}
\end{align*}

\end{document}
